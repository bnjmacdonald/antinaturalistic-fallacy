\documentclass[12pt]{article}
\usepackage[singlespacing]{setspace}
\usepackage{times}
\usepackage{tabularx}
\newcommand{\ra}[1]{\renewcommand{\arraystretch}{#1}}
\usepackage{booktabs}
\usepackage[round]{natbib}
\usepackage[labelfont=bf]{caption}
\usepackage{mathtools}
\usepackage{color}
\usepackage{subcaption}
\usepackage{graphicx}
\usepackage{placeins}
\usepackage[capposition=bottom]{floatrow}
\usepackage[margin=1.2in]{geometry}
\usepackage{chngpage}
\usepackage{multirow}
\usepackage{fancyhdr}
% \usepackage{parskip}
\usepackage{listings}
\usepackage{color}
\usepackage{xcolor}

\definecolor{dkgreen}{rgb}{0,0.6,0}
\definecolor{gray}{rgb}{0.5,0.5,0.5}
\definecolor{mauve}{rgb}{0.58,0,0.82}

\usepackage{abstract} 
\renewcommand{\abstractnamefont}{\normalfont\bfseries} 
\newcommand{\textbfit}[1]{\textbf{\textit{#1}}}
\newcommand{\todo}[1]{\textit{\textcolor{red}{$<$todo$>$ #1 $<$/todo$>$}}}
\newcommand{\question}[1]{\textit{\textcolor{red}{$<$question$>$ #1 $<$/question$>$}}}
\newcommand{\note}[1]{\textit{\textcolor{red}{$<$note$>$ #1 $<$/note$>$}}}

\newcommand{\figurepath}[1]{../output/figures/#1}

\usepackage{titlesec} 
\titleformat{\section}[block]{\LARGE\scshape\bfseries}{\thesection.}{0.4em}{} 
\titleformat{\subsection}[block]{\large\bfseries}{\thesubsection.}{0.4em}{}
\titleformat{\subsubsection}[block]{\normalsize\bfseries}{\thesubsubsection.}{0.4em}{} 
% \setlength{\parskip}{1em}

\titlespacing\section{0pt}{16pt}{0pt}
\titlespacing\subsection{0pt}{8pt}{0pt}
\titlespacing\subsubsection{0pt}{8pt}{0pt}

\pagestyle{fancy}
\fancyhf{}
\lhead{\textsc{Macdonald \& Vivalt}}
\rhead{\thepage}
% \rfoot{Page \thepage}

\usepackage{hyperref}

\hypersetup{
    colorlinks=true,
    linkcolor=blue,
    citecolor=cyan,
    urlcolor=blue
}

% to make nameref work: 
% https://tex.stackexchange.com/questions/123666/nameref-breaks-for-star-versions-of-section-paragraph-when-titlesec-is-loaded
\makeatletter
\def\ttl@useclass#1#2{%
  \@ifstar
    {\ttl@labeltrue\@dblarg{#1{#2}}}% {\ttl@labelfalse#1{#2}[]}%
    {\ttl@labeltrue\@dblarg{#1{#2}}}}
\makeatother

% --------------------------------------------------------------------- %
%                               TITLE                                   %
% --------------------------------------------------------------------- %

\title{
    \vspace{-8mm}
    \Huge\selectfont\textbf{The effectiveness of anti-naturalistic fallacy strategies on acceptance of ``clean'' meat} \\[5mm] 
    % \LARGE\selectfont\textbf{SUBTITLE}
    \vspace{5mm}
}

\author{\textbf{Bobbie NJ Macdonald} \\
        Stanford University \\
        {\href{mailto:bmacdon@stanford.edu}{bmacdon@stanford.edu}} \\
    \and \textbf{Eva Vivalt} \\
        The Australian National University \\
        {\href{mailto:eva.vivalt@anu.edu.au}{eva.vivalt@anu.edu.au}} \\
}


\date{This draft: \today{}}


\begin{document}

\maketitle
\thispagestyle{empty}

\begin{centering}
    % \vspace{-10mm}
    \Large \textsc{\textcolor{red}{Preliminary and incomplete: please do not cite}} \\
    \Large \textsc{\textcolor{red}{v0.41}} \\
    \vspace{10mm}
\end{centering}


% --------------------------------------------------------------------- %
%                             ABSTRACT                                  %
% --------------------------------------------------------------------- %


% \begin{abstract}
%     \todo{...}
% \end{abstract}


% --------------------------------------------------------------------- %
% --------------------------------------------------------------------- %
%                           INTRODUCTION                                %
% --------------------------------------------------------------------- %
% --------------------------------------------------------------------- %
% \newpage

\setcounter{page}{1}

\section{Introduction}
\label{sec:intro}

% MOTIVATION
% ----------

% CLEAN MEAT IS SUBJECT TO CONSUMER CONCERNS ABOUT "NATURALNESS" AND SAFETY
\textbf{Motivation.} ``Clean meat'' -- i.e. meat products made from cultured animal tissues -- has the potential to significantly reduce animal suffering and environmental damage while improving human health. Although clean meat products have been the source of a great deal of excitement in the media over the past few years,\footnote{See, e.g. \href{https://www.washingtonpost.com/news/wonk/wp/2015/05/20/meet-the-future-of-meat-a-10-lab-grown-hamburger-that-tastes-as-good-as-the-real-thing/}{https://www.washingtonpost.com/news/wonk/wp/2015/05/20/meet-the-future-of-meat-a-10-lab-grown-hamburger-that-tastes-as-good-as-the-real-thing/}; \href{http://fortune.com/2016/02/02/lab-grown-memphis-meats/}{http://fortune.com/2016/02/02/lab-grown-memphis-meats/}; \href{http://gizmodo.com/the-future-will-be-full-of-lab-grown-meat-1720874704}{http://gizmodo.com/the-future-will-be-full-of-lab-grown-meat-1720874704}.} these products are likely to face substantial public resistance since they conflict with prevailing cultural intuitions and cognitions. Since clean meat products are viewed by many potential consumers as ``artificial'' or ``lab-grown'' meat, these products conflict with the widespread heuristic that ``what is natural is good''. Moreover, the expected benefits of clean meat are either temporally distant (e.g. long-term health benefits, avoiding catastrohpic climate change) or spatially removed (e.g. animal suffering in factory farms), making it difficult for consumers to incorporate these benefits in their decisionmaking calculus.\footnote{For a review of recent research on individual decisions with temporally or spatially distant consequences, see \citet{Wade-Benzoni2009,Markowitz2012}.}

% As a result, many consumers are likely to feel uncertain about the health and safety implications of clean meat, and have trouble appreciating the environmental and animal welfare benefits that clean meat products can provide.

% THERE IS LITTLE RESEARCH ON EFFECTIVE MESSAGES FOR ADDRESSING THESE CONCERNS
The ``naturalistic heuristic'' -- where consumer attitudes are biased towards products that are more congruent with their notion of what is ``natural'' for humans to consume and what kinds of organisms/chemicals/processes occur in the natural environment -- is not unique to clean meat. Genetically modified (GM) foods, vaccines, nuclear power, and stem cell research face similar barriers to acceptance in the general public. In short, although a naturalistic heuristic may in general help consumers choose healthier food products with less added sugars, fats, and preservatives, at the same time this heuristic poses a significant barrier to public acceptance of technologies that could have far-reaching benefits for scientific advancement, public health, environmental sustainability, and animal welfare.

There is a rapidly growing body of research on public perceptions of vaccines, GM foods, and nuclear power \citep[see, e.g.,][]{Yaqub2014}. which has recently expanded to the study of public attitudes towards clean meat products \citep[e.g.][]{Wilks2017}. Yet, few studies on vaccines, GM foods, or nuclear power have examined the effectiveness of competing messaging strategies at overcoming these naturalistic concerns and increasing consumer acceptance.\footnote{For exception, see \citet{Nyhan2015,Nyhan2014}.} In particular, we are not aware of any experimental research that compares the efficacy of different messaging strategies for increasing consumer acceptance of clean meat products. Given that clean meat products are expected to become widely available to consumers in the next couple of years, this lack of research on effective strategies for overcoming consumers' naturalistic heuristic raises important concerns about the likely acceptance and uptake of clean meat products.

% and have received major investments from food companies such as Tyson Foods

% Moreover, this line research has extended to clean meat products, which are increasingly projected to become the ``future of meat''\footnote{e.g. https://www.washingtonpost.com/news/wonk/wp/2015/05/20/meet-the-future-of-meat-a-10-lab-grown-hamburger-that-tastes-as-good-as-the-real-thing/}

% There is a great deal of non-experimental research on public acceptance of GMOs, vacinnes, and similar products/technologies \citep[see, e.g.][]{}, yet 

% How do individuals respond when exposed to this ``naturalistic'' reasoning? And how can we overcome this naturalistic fallacy for technologies that offer significant welfare improvements?

% SUMMARY OF RESEARCH DESIGN
% --------------------------
\textbf{Research design.} In this study, we set out to answer three main questions: (1) To what extent is consumer acceptance of clean meat products driven by concerns about health, safety, and ``naturalness'' -- each of which is indicative of the naturalistic heuristic -- rather than more conventional consumer concerns about cost and taste? (2) As a novel consumer product, how susceptible is consumer acceptance of clean meat products to \textit{negative social information}, consisting of negative reactions towards clean meat products from peers? And (3) how effective is ``direct debunking'' of the naturalistic heuristic at increasing consumer acceptance of clean meat products compared to a simple ``social norm'' pro-clean meat message and a placebo message?

To address these questions, we conduct a three-wave 2x4 full factorial survey experiment that examines how messaging strategies and \textit{negative social information} affect individual acceptance of clean meat products and susceptibility to the naturalistic fallacy. First, we randomly assign participants to receive negative social information or not, consisting of a sample of five negative reactions towards clean meat from previous survey respondents (e.g. ``Artificial meat sounds disgusting''). Second, we randomly assign participants to read either a \textit{placebo} article or one of three pro-clean meat articles: a \textit{natural does not mean good} appeal, \textit{most foods are unnatural} appeal, or \textit{social norm} appeal.

The first two appeals are variations of a ``direct debunking'' strategy, similar to the ``misperception correction'' messaging strategy commonly used to combat anti-vaccination beliefs.\footnote{See, for instance: \href{http://www.who.int/vaccine_safety/initiative/detection/immunization_misconceptions}{www.who.int/vaccine\_safety/initiative/detection/immunization\_misconceptions}; \href{https://www.health.ny.gov/prevention/immunization/vaccine_safety/misperceptions}{www.health.ny.gov/prevention/immunization/vaccine\_safety/misperceptions}.} Both the \textit{natural does not mean good} and \textit{most foods are unnatural} appeals invokes the naturalistic heuristic in order to argue against its application in the case of clean meat products, although the two appeals differ in emphasis. Specifically, the \textit{natural does not mean good} appeal focuses narrowly on debunking the naturalistic fallacy, pointing out that some seemingly ``natural'' compounds are clearly bad (e.g. arsenic) while other seemingly ``unnatural'' products clearly have positive benefits (e.g. antibiotics). In contrast, the \textit{most foods are unnatural} appeal emphasizes how clean meat products are similar to many other seemingly ``unnatural'' foods that have already been widely accepted by most consumers, thereby encouraging readers to add clean meat products to the set of products they deem as acceptable. This messaging strategy is meant to leverage consumers' ``cultural intuitions'' \citep{Miton2015} about what is acceptable by equating clean meat products with other products they have already accepted. Finally, the \textit{social norm} appeal does not make any persuasive arguments invoking the naturalistic falalcy, instead conveying a \textit{descriptive} norm that many consumers are excited about clean meat and would like to try it once it becomes available in their area.

% Building on recent work examining the effectiveness of messaging strategies for improving public acceptance of vaccines \citep[e.g.][]{Nyhan2014}, in this study we examine how messages and social information affect individual acceptance of clean meat products and susceptibility to the naturalistic fallacy. Specifically, we implement a 4x2 full factorial survey experiment with three different anti-naturalistic messages (and a placebo message) crossed with whether or not subjects are exposed to social information beforehand (i.e. how previous survey respondents have reacted to clean meat).

% SUMMARY OF MAIN FINDINGS
% ------------------------
\textbf{Results.} We produce three main sets of results.
% WARINESS DRIVEN BY CONCERNS ABOUT HEALTH AND NATURALNESS
\underline{First}, before turning to the experimental results, we show that consumers' wariness about clean meat is driven far more by concerns about whether clean meat is natural, safe, and healthy -- which are symptoms of the naturalistic fallacy -- than how clean meat products taste or how much they cost.
% PERSISTENT EFFECTS OF NEGATIVE SOCIAL INFORMATION
\underline{Second}, we show that even small amounts of negative social information about clean meat can make consumers significantly more wary about clean meat products, and that these effects persist for a period of at least 10 weeks. These effects are strongest among individuals who already have low levels of interest in clean meat.
% LACK OF PERISTENT APPEAL EFFECTS
\underline{Third}, we show that, while pro-clean meat appeals based on debunking the naturalistic fallacy or conveying social norms can counteract nearly all of the negative effects of negative social information \textit{in the short term}, the effects of the pro-clean meat appeals we evaluate in this study are indistinguishable from zero after 10 weeks. There is \textit{some} evidence that the \textit{most foods are unnatural} appeal -- which combines debunking with familiarity to common foods -- led to persistent increases in interest in clean meat products among individuals who had low levels of interest in clean meat at baseline. Overall, however, we fail to find any systematic evidence of persistent effects from the debunking and social norm appeals used in this study.
% In particular, these appeals produce substantial improvements in consumer attitudes towards clean meat and effectively reduce concerns that clean meat is ``unnatural''. Nevertheless, we find little evidence that the direct debunking appeals are any more effective than a simple social norm message. 
% HETEROGENEITY IN APPEAL EFFECTS
% \underline{Fourth}, we provide some evidence that negative social information has the strongest effects on individuals who already have low levels of interest in clean meat. Conversely, we find some evidence that the \textit{most foods are unnatural} appeal 
% COMBATING NEGATIVE SOCIAL INFORMATION
% \underline{Fifth}, we provide some evidence that the XXXX appeal is more effective than the other two appeals at counteracting the effects of negative social information.

% positive effects of these appeals are \textit{not} restricted to individuals who were already highly supportive of clean meat products or who already consumed very few servings of conventional meat products per week.

% First, we show that concerns about clean meat being ``unnatural'' and ``unsafe'' -- rather than typical consumer concerns about price and taste -- are overwhelmingly responsible for consumer inhibition about clean meat products.

% First, we show that all three of the pro-clean meat appeals are moderately effective at increasing consumer acceptance of clean meat in the short term. In particular, the \textit{social norms} message results in the largest increase in acceptance of clean meat, perceived safety of clean meat, and perceived health benefits of clean meat. On the other hand, respondents were much \textit{less} likely to cite lower cost as a possible benefit of clean meat, suggesting that concerns about the safety and health of clean meat have been displaced to other concerns.

% Second, we show that receiving small amounts of negative social information is a significant impediment to consumer acceptance of clean meat, largely washing out the positive effects of the pro-clean meat messages. This is especially concerning since the negative social information used in this study was a ``weak'' treatment, containing only five short sentences from completely anonymous survey respondents.

% Third, although we find evidence that the treatment appeals remain effective in the face of negative social information, the pernicious effects of negative social information outweigh the positive effects of any of the three treatment appeals. In other words, none of the three treatment appeals fully counteract the negative effects of social information about clean meat products.

% Fourth, we show that \todo{insert finding on who is most susceptible to naturalistic fallacy}.

% BROADER SIGNIFICANCE
% --------------------
\textbf{Contributions.} The purpose of this study is to shed light on the ways in which individuals respond to the naturalisic fallacy and whether information treatments can help to counter it. More broadly, this research deepens our understanding of the ways in which individuals form opinions towards new technologies that may conflict with cultural intuitions. Given that online information and social media are an important source of misinformation that drives negative attitudes towards many technologies and products (e.g. GM foods, vaccines, nuclear power), these findings offer promise that simple online articles can effectively counteract negative attitudes.

% Since individuals acquire a great deal of their information online, finding effective online messaging strategies for combating individual fallacies and negative attitudes is increasingly important.


% --------------------------------------------------------------------- %
% --------------------------------------------------------------------- %
%                          EXISTING WORK                                %
% --------------------------------------------------------------------- %
% --------------------------------------------------------------------- %


% \section{Existing work}
% \label{sec:existing-work}

% \todo{...}

% % EFFECTS OF CORRECTING MIS-INFORMATION


% Resistance to scientific evidence literature.

% Correcting information can make things worse (backfire effect). e.g. via motivated reasoning and coming up with additional refutations.

% % 
% Framing as gains vs. avoided losses. Is this literature relevant?


% ``similar to foods'' attempts to overcome status quo bias (omission vs. commission)?


% how to spread counter-intuitive beliefs?

% Animal suffering and environmental degradation are far removed from consumers' everyday experiences.

% --------------------------------------------------------------------- %
% --------------------------------------------------------------------- %
%                        EXPERIMENTAL DESIGN                            %
% --------------------------------------------------------------------- %
% --------------------------------------------------------------------- %

\section{Experimental design}
\label{sec:design}

\textbf{Research questions.} In this study, we set out to answer three main questions. \underline{First}, to what extent is consumer acceptance of clean meat products driven by concerns about health, safety, and ``naturalness'' -- each of which is indicative of the naturalistic fallacy -- rather than more conventional consumer concerns about cost and taste? \underline{Second}, as a novel consumer product, how susceptible is consumer acceptance of clean meat products to ``negative social information''? \underline{Third}, how effective is ``direct debunking'' of the naturalistic heuristic at increasing consumer acceptance of clean meat products compared to a simple ``social norm'' pro-clean meat message and a placebo message?
% \item Are ``direct debunking'' messages more effective at increas
% \item What kinds of messages are most effective at combating the naturalistic fallacy and increasing individual acceptance of clean meat in the face of negative social information about clean meat?
% \item Absent of any messaging or social information, what kinds of individuals are most susceptible to bringing up the naturalistic fallacy after learning about clean meat?


% \begin{enumerate}
%     \item What kinds of messages are effective at combating the naturalistic fallacy and increasing individual acceptance of clean meat?
%     \item How does negative social information affect acceptance of clean meat and susceptibility to the naturalistic fallacy?
%     \item What kinds of messages are most effective at combating the naturalistic fallacy and increasing individual acceptance of clean meat in the face of negative social information about clean meat?
%     \item Absent of any messaging or social information, what kinds of individuals are most susceptible to bringing up the naturalistic fallacy after learning about clean meat?
% \end{enumerate}

% DATA COLLECTION
\textbf{Data collection.} Data was collected in three online survey waves.

% BASELINE WAVE
\textit{Baseline survey (wave 1).} First, participants were asked about demographics, current levels of meat consumption, attitudes, and potential moderators. All participants were also given basic information about clean meat and some purported environmental/health/ethical benefits of consuming clean meat products.

% TREATMENT WAVE
\textit{Treatment exposure (wave 2).} Second, approximately one week after completing the baseline survey, the same participants were recontacted and asked to complete a second survey. Participants were randomly assigned to one of eight experimental cells (see below). Then participants assigned to the four ``social information'' cells were shown a page containing five short quotes from previous survey respondents that contain negative sentiment about clean meat (e.g. ``This seems very unnatural. I don't feel comfortable about this.''). All participants were then shown a placebo news article or one of three pro-clean meat appeals corresponding to their experimental cell. Immediately afterwards, all participants completed a short survey containing a discrete choice block (see below) and several open-ended questions regarding their reactions to the news article. Participants were also asked for their attitudes towards clean meat, willingness-to-pay, and interest in further information about clean meat products and vegetarian products.

% ENDLINE WAVE
\textit{Endline survey (wave 3).} Finally, approximately 10 weeks after completing the treatment exposure survey, the same participants were recontacted and asked to complete a followup survey. Participants were asked to complete a short survey containing a discrete choice block, attitudes towards clean meat, willingness-to-pay, and interest in further information about clean meat products and vegetarian products.

% EXPERIMENTAL CONDITIONS
\textbf{Experimental conditions.} This study is organized as a randomized 2x4 full factorial design, examining how social information and messaging appeals affect individual acceptance of clean meat products and susceptibility to the naturalistic fallacy. First, we randomly assign participants to receive \textit{negative social information} or not, consisting of a sample of five negative reactions from previous survey respondents towards clean meat (e.g. ``Artificial meat sounds disgusting''). 

% A screenshot of this negative social information is provided in the appendix \todo{}.

Second, we randomly assign participants to read one of four articles: a \textit{placebo} message, \textit{natural does not mean good} appeal (appeal \#1), \textit{most foods are unnatural} appeal (appeal \#2), or \textit{social norm} appeal (appeal \#3). All messages are approximately 150-200 words in length, with three images that help to convey the main message. The \textit{placebo} message urges participants to walk more, and makes no mention of clean meat products or meat consumption.

% Screenshots are provided in the appendix \todo{}

Appeals 1 and 2 are variations of a ``direct debunking'' strategy, similar to the ``misperception correction'' messaging strategy commonly used to combat anti-vaccination beliefs. While this messaging strategy is in widespread use, there is very little evidence that such correction-oriented appeals are effective at countering negative attitudes/beliefs \citep{Nyhan2015,Nyhan2014}. While both the \textit{natural does not mean good} and \textit{most foods are unnatural} appeals invoke the naturalistic heuristic in order to argue against its application in the case of clean meat products, the two appeals differ in emphasis. The \textit{natural does not mean good} appeal (appeal \#1) provides several examples of objects/phenomena that are clearly good but unnatural (e.g. antibiotics) and objects/phenomena that are clearly bad but natural (e.g. appendicitis). In contrast, the \textit{most foods are unnatural} appeal (appeal \#2) describes how nearly all foods we eat today have been artificially engineered through selective breeding and other practices such that they no longer resemble their naturally occuring ancestors. This appeal is designed to more directly align clean meat products with consumers' intuitions about whether clean meat is likely to be beneficial to their health, thereby making it easier to accept the ``counter-intuitive'' conclusion that clean meat is a desirable product despite the naturalistic heuristic.\footnote{For a discussion of how ``culturally shared intuitions'' -- such as the naturalistic heuristic -- may affect beliefs towards vaccines and similar technologies, see \citet{Miton2015}.} Specifically, by emphasizing how clean meat is similar to many other kinds of foods that have been widely accepted by most consumers, readers are encouraged to add clean meat products to the set of products that are deemed acceptable rather. Hence, if we were to find that the \textit{most foods are unnatural} appeal is more effective than the \textit{natural does not mean good}, this would suggest that appealing to consumers' cultural intutions is an important component of new product acceptance.

Finally, the \textit{social norm} appeal (appeal \#3) conveys a \textit{descriptive} norm that many consumers are excited about clean meat and would like to try it once it becomes available in their area. This appeal makes no attempt to debunk the naturalistic heuristic or to emphasize the benefits of clean meat. Instead, it merely signals to readers that many other consumers seem to be excited about clean meat products, rather than concerned about potential health and safety implications.

\textbf{Subject recruitment and sample size.} We recruited participants through Amazon Mechanical Turk (MTurk). Each participant was paid US\$0.50 for completing the baseline survey, US\$0.50 for completing the treatment survey, and US\$0.50 for completing the endline survey (for a total of US\$1.50 for participation in the entire study). Following the baseline survey, we recontacted participants via email. We recruited 400 subjects per experimental cell, for a total of 3200 subjects.

% Previous research on MTurk with similar multi-wave designs have yielded retention rates around 80\% between baseline and endline, which would leave us with approximately 320 subjects per cell.

% We will use block randomization based on baseline survey responses in order to increase statistical power.

\textbf{Primary outcome measures.} All variables described in this section were measured in the baseline, treatment, and endline survey waves, unless otherwise stated. For all analyses reported in Section \ref{sec:results}, non-binary dependent variables are standardized to have mean equal to zero and variance equal to one.

% Full details on all questions are provided in the supplementary materials.

\textit{Interest in clean meat.} We collect several attitudinal measures on attitudes towards clean meat, such as ``how interested are you in purchasing the clean meat product you just read about?'' (1-7 scale) and ``Would you like to be notified when clean meat products are available in your area?'' (yes/maybe/no and provision of e-mail address in a follow-up question asked to those answering ``yes'' or ``maybe''). Respondents were only asked in waves 1 and 2 if they would like to be notified when clean meat products become available in their area.

\textit{Concerns about clean meat.} We asked participants to select the two most important concerns they have about clean meat products. We also provided participants with an open-ended text box to state their most important concerns about clean meat products.

\textit{Perceived benefits of clean meat.} We asked participants to select the benefits they think clean meat products will have. We also provided participants with an open-ended text box to state what they perceive will be the most important benefits of clean meat products.

\textit{Willingness to pay for clean meat.} We infer participants’ willingness to pay for clean meat from a discrete choice experiment at the end of the treatment survey. Respondents were presented with sets of descriptions of two or three different products, each consisting of a set of 2 attributes: (a) Product: clean meatballs, vegetarian meatballs, conventional meatballs; (b) Price per lb: \$5, \$10, \$15, or \$20. A full factorial design accounting for all interactions among those exposed to information about clean meat consists of $12$ different combinations (3 products $\times$ 4 prices). We asked respondents to answer one of two alternative blocks of 6 questions (randomly assigned).

\textbf{Secondary outcome measures.} All variables described in this section were measured in the baseline, treatment, and endline survey waves, unless otherwise stated.

\textit{Attitudes towards factory farming.} We collected four attitudinal measures on meat consumption and factory farming. Participants were asked to rate whether and how much factory farming contributes to animal suffering and whether this is an issue they care about; whether and how much factory farming contributes to environmental degradation, and whether this is an issue they care about; and whether they think it is morally preferable to avoid eating factory farmed meat. Participants were also asked whether they would be interested in receiving tips on how to reduce their meat consumption.

\textit{Perceptions of social norms.} On a seven point scale, participants were asked whether they agree or disagree with the statement that more and more people in the US are reducing their meat consumption (1=strongly disagree, 7=strongly agree).

\textit{Perceptions of vegetarians.} Participants were asked to give their feelings towards vegetarians (1=extremely positive, 7=extremely negative). 

\textit{Perceptions of intelligence and sentience.} Participants were asked to rate seven species of animals on a 1-7 scale in terms of perceived intelligence (1=very unintelligent, 7=very intelligent). Similarly, participants were asked to rate how capable these seven species of animals were of experiencing pain and suffering on a 1-7 scale (1=completely incapable, 7=highly capable). We use ``humans'' as a comparison group in the analyses.

\textit{Ease of reducing meat consumption.} On a seven point scale ranging from very difficult (1) to very easy (7), participants were asked to rate how easy it would be to completely eliminate conventional meat products from their diet over the next year and how easy it would be to reduce their consumption of conventional meat products by 25\% over the next year.


% --------------------------------------------------------------------- %
% --------------------------------------------------------------------- %
%                             SECTION                                   %
% --------------------------------------------------------------------- %
% --------------------------------------------------------------------- %


\section{Results}
\label{sec:results}

% In the results reported below, the average treatment effect (ATE) is estimated by regressing each of the main outcome measures (described above) on treatment group membership.

% Todo: To reduce the influence of outliers, we report all analyses after removing the top 2.5\% and bottom 2.5\% of responses (\todo{}). Further, all individuals who say at baseline they are vegetarian or vegan are dropped from the sample (\todo{}). To improve power, we control for baseline meat consumption and the Utilitarian score (described elsewhere) where this is significant (\todo{}). To address concerns about multiple hypothesis testing, we restrict the false discovery rate (FDR) using the weighted FDR control method proposed in \citet{Benjamini1997} (\todo{}).

Overall, how interested are consumers in clean meat products? Based on the results of our survey, there is a sizable minority of consumers who are interested in trying clean meat products. At the end of the treatment survey wave (wave 2), 36.4\% of respondents in the control group ($n=308$) entered their email address in order to be notified when clean meat products become available in their area. In addition, 51.3\% of respondents answered ``probably yes'' or ``definitely yes'' to whether they would eat a clean meat product, while 28.6\% of respondents said that they were ``very interested'' or ``extremely interested'' in purchasing clean meat products. See Figure \ref{fig:interest} in the \nameref{sec:appendix} for further information on these descriptives.

In the analyses that follow, we examine consumer attitudes towards clean meat products in several stages. \underline{First}, before turning to the experimental results, we show that consumers' wariness about clean meat is driven far more by concerns about whether clean meat is natural, safe, and healthy -- which are symptoms of the naturalistic fallacy -- than how clean meat products taste or how much they cost.
% PERSISTENT EFFECTS OF NEGATIVE SOCIAL INFORMATION
\underline{Second}, we show that even small amounts of negative social information about clean meat can make consumers significantly more wary about clean meat products, and that these effects persist for a period of at least 10 weeks. These effects are strongest among individuals who already have low levels of interest in clean meat.
% LACK OF PERISTENT APPEAL EFFECTS
\underline{Third}, we show that, while pro-clean meat appeals based on deunking the naturalistic fallacy or conveying social norms can counteract nearly all of the negative effects of negative social information \textit{in the short term}, the effects of the pro-clean meat appeals we evaluate in this study are indistinguishable from zero after 10 weeks. There is \textit{some} evidence that the \textit{most foods are unnatural} appeal -- which combines debunking with familiarity to common foods -- led to persistent increases in interest in clean meat products among individuals who had low levels of interest in clean meat at baseline. Overall, however, we fail to find any systematic evidence of persistent effects from the debunking and social norm appeals used in this study,


% \underline{First}, we show that consumers' wariness about clean meat is driven far more by concerns about whether clean meat is natural, safe, and healthy than how it tastes and how much it costs. \underline{Second}, we show that even small amounts of negative social information about clean meat can make consumers significantly more wary about clean meat products, making the task of marketing clean meat products even more difficult. \underline{Third}, on a more optimistic note, we show that pro-clean meat appeals which debunk the naturalistic fallacy can counteract nearly all of the negative effects of negative social information. In particular, these appeals produce substantial improvements in consumer attitudes towards clean meat and effectively reduce concerns that clean meat is ``unnatural''. Nevertheless, we find little evidence that the direct debunking appeals are any more effective than a simple social norm message. \underline{Fourth}, we show that the positive effects of these appeals are \textit{not} restricted to individuals who were already highly supportive of clean meat products or who already consumed very few servings of conventional meat products per week.



% ATTITUDINAL BARRIERS TO INTEREST IN CLEAN MEAT
% ----------------------------------------------
\subsection{Naturalistic reasoning undermines interest in clean meat}

% \note{We don't directly measure }

How important are consumer concerns emanating from the naturalistic fallacy -- such as perceptions that clean meat is ``unnatural'', ``unsafe'', or ``unhealthy'' -- in undermining willingness to eat clean meat products relative to more conventional consumer concerns, such as price and taste? Here, we show that the former concerns are the main barriers to consumer interest in purchasing and eating clean meat products, while price and taste are only weakly related to consumer interest. Figure \ref{fig:barrier_concerns} illustrates this claim, displaying linear regression coefficients from regressing the change in respondent $i$'s interest in clean meat between waves 1 and 3 (measured by \textit{feel}, \textit{interest purchase}, and \textit{would eat}) on the change in whether respondent $i$ expressed a particular concern about clean meat. We examine five core concerns here: taste, cost, safety, unhealthy, unnatural.

As shown, respondents who went from unconcerned to concerned about clean meat's ``naturalness'' between waves 1 and 3 became substantially less interested in clean meat products over the same time period.
% FOOTNOTE
\footnote{Or, conversely, respondents who went from being concerned to unconcerned about clean meat's ``naturalness'' between waves 1 and 3 became substantially more interested in clean meat products over the same time period. The results in this figure average across these two types of respondents -- that is, those who went from concerned to unconcerned and unconcerned to concerned -- as well as respondents who did not change their concern between waves 1 and 3.}
% END FOOTNOTE
Specifically, becoming concerned about whether clean meat is natural is associated with a 0.35 standard deviation decrease in \textit{would eat} ($p < 0.01$) and a 0.23 standard deviation decrease in \textit{would purchase}  ($p < 0.01$). These patterns are similar for concerns about the safety and health effects of clean meat products. In contrast, changes in concerns about taste and cost of clean meat products are only weakly associated with interest in clean meat.
% FOOTNOTE
\footnote{If anything, increases in concerns about cost and taste are \textit{positively} associated with interest in clean meat products. This positive association likely flows from the fact that respondent concerns were measured in the form of a multichoice checkbox question. Respondents with strong anti-clean meat feelings focused on the safety, health effects, and naturalness of clean meat, crowding out attention to cost and taste. Respondents with pro-clean meat feelings tended to only check off taste and/or cost for a lack of alternative concerns, thereby inducing the positive relationship we observe.}
% END FOOTNOTE

Furthermore, among respondents who did \textit{not} raise the ``unnatural'' concern about clean meat in the baseline survey wave, 44.7\% provided an email address at the end of the treatment wave to be notified when clean meat becomes available in their area. In contrast, only 28.9\% of respondents who listed ``unnatural'' as a concern provided an email address.\footnote{This strong association between the ``unnatural'' concern and interest in clean meat persists even after controlling for other concerns and demographics in a simple linear probability model framework.} By comparison, 41.2\% of respondents who did \textit{not} raise ``taste'' as a concern entered an email address in contrast to 32.4\% among respondents raising the ``taste'' concern. 

% Hence, while concerns about the extent to which clean meat is natural, safe, and healthy were raised \textit{less} often than concerns about price and taste, the former concerns are much more strongly associated with individual interest in clean meat products.

% These findings hold when controling for other concerns raised and demographic factors.

% ------------------------------------- %
% FIGURE:
\begin{figure}[h]
    \begin{center}
    \includegraphics[width=0.7\linewidth]{\figurepath{barrier_concerns.png}}
    \end{center}
    \caption{\label{fig:barrier_concerns}\footnotesize \textbf{Effect of concerns about clean meat on interest in clean meat}. Displays the estimated effect of a change in concerns about clean meat between waves 1 and 3 on corresponding change in interest in clean meat. Horizontal bars represent 90\% and 95\% confidence intervals. Non-binary dependent variables are standardized to have mean equal to zero and variance equal to one.}
    % Average interest in clean meat (measured in treatment wave) among respondents raising each of five possible concerns about clean meat (measured at baseline), showing that ``unnatural'', ``unhealthy'', and ``safe'' concerns are associated with large reductions in interest in clean meat on average. Means are displayed separately for individuals expressing the concern and individuals not expressing the concern. Means are computed from the subset of respondents who were not exposed to any treatment condition ($n=308$). Horizontal bars represent 90\% and 95\% confidence intervals.}
    % \subcaption*{\small \textbf{Fig. \ref{fig:2A}.} }
\end{figure}
% ------------------------------------- %
% \FloatBarrier

% TRENDS IN CONCERNS BETWEEN WAVES
Given the novelty of clean meat products, these concerns about whether clean meat is natural and safe might merely be the result of an initial ``shock factor'', such that consumers' concerns would shift more towards price and taste as they become more comfortable over time with the normalcy of clean meat products. Our results offer some support for this perspective. Looking just at respondents in the control group who completed all three survey waves ($n=174$), the percentage of respondents raising ``unnatural'' as a concern dropped from 55.2\% to 48.3\% between the baseline and endline waves (diff: 6.9\%; $p = 0.09$). Similarly, the percentage of respondents raising ``unsafe'' as a concern dropped from 51.1\% to 43.7\% (diff: 7.5\%; $p = 0.08$). However, there was only a slight drop in respondents raising ``unhealthy'' as a concern, from 11.5\% to 10.3\% (diff: 1.2\%; $p = 0.66$). Conversely, the proportion of respondents citing ``cost'' as a concern increased moderately from 60.3\% of respondents to 66.1\% (diff: 5.8\%; $p = 0.15$), while the proportion of respondents raising ``taste'' as a concern decreased from 57.5\% to 51.1\% (diff: 6.5\%; $p = 0.12$). Hence, while we find some evidence that consumer concerns about clean meat shift away from ``unnatural'' and ``unsafe'' and move towards price through repeated exposure (without any further information about the benefits/costs of clean meat), concerns clean meat's health effects appear more sticky.

% NOTE: do we have a comparison group of individuals (e.g. from another study) who were only exposed once to clean meat? This would be useful to nail down the effect of repeated exposure.


% SOCIAL INFORMATION EFFECTS
% --------------------------
\subsection{Effects of Negative social information}

% MOTIVATION
How contagious are anti-clean meat attitudes? As we've shown above, a sizable minority of consumers have a positive orientation towards clean meat products, yet at the same time many consumers are wary about the safety, naturalness, and health consequences of these products. Given the novelty of clean meat products in the minds of consumers, however, the marginal effects of new information are likely to be particularly high. As a result, we should expect these attitudes to fluctuate widely as consumers become exposed to different clean meat products, marketing campaigns, and media narratives.

% DESCRIPTION OF TREATMENT
Here, we examine the extent to which the naturalistic fallacy is contagious. Specifically, we examine the extent to which a small amount of negative social information -- in the form of negative sentiment towards clean meat expressed by respondents from a previous survey -- undermines consumer interest in clean meat products. As described above, respondents in this study were randomly assigned to be exposed to negative social information at the beginning of the treatment wave, consisting of five short quotes from respondents to a previous survey expressing negative sentiment towards clean meat (e.g. ``Our guts are not meant to digest unnatural things'').

% DESCRIPTION OF EFFECTS ON DISCRETE CHOICE
The top half of Figure \ref{fig:dce_effects} displays the effects of negative social information on respondents' stated preference for clean meat products based on the discrete choice experiment. In the panels titled ``Clean meatballs'' and ``Vegetarian meatballs'', points in black represent the average difference in the probability of choosing the product compared to ``conventional meatballs''. In panels titled ``\$5 higher cost'', points in black represent the average difference in the probability of choosing a product that is \$5 more expensive per pound. Whether or not respondents were exposed to negative social information, they were significantly less likely to say they would purchase clean meatballs relative to conventional meatballs. However, respondents exposed to negative social information expressed an even stronger preference for conventional meatballs over clean meatballs that unexposed participants. In wave 2, exposed participants were approximately 16 percentage points less likely to choose meatballs when they were ``clean'' rather than ``conventional'', compared to a difference of only 7.7 percentage points among unexposed participants (diff: 8.3; $p < 0.01$). These differences also emerged in wave 3, although at a substantially smaller scale -- such that ``clean'' meatballs were 21 percentage points less likely to be selected than ``conventional'' among exposed respondents, compared to 18.1 percentage points less likely among unexposed respondents (diff: 2.9; $p < 0.05$).

% TODO: say something positive about greater willingness to try clean meat than veg meats.

% DESCRIPTION OF EFFECTS ON INTEREST AND CONCERN
Negative social information also had strong effects on respondents' self-reported interest in clean meat products and concerns about clean meat products. The top panel of Figure \ref{fig:effects} displays the effects of negative social information on attitudes towards clean meat, comparing all respondents who were exposed to the negative quotes against unexposed respondents. Points in black represent the difference in means between exposed and unexposed respondents on changes in interest in clean meat and concerns about clean meat between waves 1 and 3. Gray points represent the differences in means between waves 1 and 2. The results are discouraging, showing that small amounts of social information from complete strangers significantly decreases respondents' interest in clean meat products, even when measured 10 weeks after treatment exposure. For instance, exposed respondents reported a significant decrease in willingness to try clean meat products (\textit{would eat}: $d=-0.31$; $p < 0.01$) and interest in purchasing meat products (\textit{interest purchase}: $d=-0.32$; $p < 0.01$) between waves 1 and 2 relative to unexposed respondents. In addition, negative social information led to increases in concerns about the health effects (\textit{concern unhealthy}: $d=0.07$; $p < 0.01$) and safety (\textit{concern unsafe}: $d=0.06$; $p < 0.01$) of clean meat products between waves 1 and 2.

Despite some attenuation, these effects persist when examining the corresponding changes between waves 1 and 3 (10 weeks after treatment). For instance, even after 10 weeks, exposed respondents reported a significant decrease in willingness to try clean meat products (\textit{would eat}: $d=-0.17$; $p < 0.01$) and interest in purchasing meat products (\textit{interest purchase}: $d=-0.13$; $p < 0.05$) between waves 1 and 2 relative to unexposed respondents. Exposed respondents also remained more likely to report concerns about the health effects of clean meat products (\textit{concern unhealthy}: $d=0.04$; $p < 0.09$). Exposed respondents were also somewhat more likely than unexposed respondents to report concerns about whether clean meat is natural (\textit{concern unnatural}; $d=0.04$; $p = 0.2$), although this difference does not reach conventional levels of statistical significance.

These results are disconcerting, showing that even very weak forms of negative social information -- as in the form of five short quotes from completely anonymous individuals -- can lead to persistent anti-clean meat attitudinal shifts. Even worse, these attitudinal shifts persist despite the fact that most respondents were asked to read a pro-clean meat appeal immediately after exposure to negative social information.
% FOOTNOTE
\footnote{Except for those exposed to the control appeal, representing approximately 25\% of respondents.}
% END FOOTNOTE
In the next subsection, we examine the effectiveness of these pro-clean meat appeals at countering negative social information.

% In particular, negative social information information led to a decrease of 3.4\% percentage points ($p < 0.05$) in the proportion of individuals who entered their email address to be notified when clean meat products are available in their area.

% In addition, negative social information led to a significant increase in the proportion of respondents concerned about whether clean meat is healthy and safe (\ref{fig:effects_social}, panel 1). Exposed individuals were 7.5\% percentage points ($p < 0.01$) more likely to raise ``unhealthy'' as a concern about clean meat than individuals not exposed to negative social information, as well as 5.0\% percentage points more likely to raise ``unsafe'' as a concern.

% These results show that even small amounts of negative social information wash out -- and in some cases overwhelm -- the positive effects of the treatment appeals. In other words, respondents who were not exposed to either treatment condition expressed no more interest in clean meat than respondents who were exposed to negative social information and any one of the three appeals.

% % ------------------------------------- %
% % FIGURE:
% \begin{figure}[h]
% \begin{subfigure}{0.48\textwidth}
%     \centering
%     \subcaption{Wave 2}
%     \includegraphics[width=1\linewidth]{\figurepath{effects_chg_1_2.png}}
% \end{subfigure}
% \begin{subfigure}{0.48\textwidth}
%     \centering
%     \subcaption{Wave 3}
%     \includegraphics[width=1\linewidth]{\figurepath{effects_chg_1_3.png}}
% \end{subfigure}
% \caption{\small\label{fig:effects}\textbf{Social information and appeal effects, wave 2 vs. wave 3.} Each black dot represent the estimated treatment effect of negative social information on a single outcome measure. Outcome measures are computed as the change between baseline and treatment waves. Horizontal bars represent 90\% and 95\% confidence intervals. For comparison, the small colored dots represent estimated treatment effects of the three treatment appeals. Treat article \#1: \textit{natural does not mean good} appeal; Treat article \#2: \textit{most foods are unnatural} appeal; Treat article \#3: \textit{social norm} appeal.}
% \end{figure}
% % ------------------------------------- %


% ------------------------------------- %
% FIGURE:
\begin{figure}[h]
    \begin{center}
    \includegraphics[width=0.7\linewidth]{\figurepath{dce_effects.png}}
    \end{center}
    \caption{\label{fig:dce_effects}\footnotesize \textbf{Effects on discrete choice.} Displays the differences between treatment conditions in respondents' willingness to select ``conventional meatballs'' vs. ``clean meatballs'' vs. ``vegetarian'' meatballs at different price points in a discrete choice experiment. In panels titled ``Clean meatballs'' and ``Vegetarian meatballs'', points in black represent the average difference in the probability of choosing the product compared to ``conventional meatballs''. In panels titled ``\$5 higher cost'', points in black represent the average difference in the probability of choosing a product that is \$5 more expensive per pound. Y-axis indicates treatment condition. Black points represent estimates from wave 3, gray points represent estimates from wave 2. Horizontal bars represent 90\% and 95\% confidence intervals. Treat article \#1: \textit{natural does not mean good} appeal; Treat article \#2: \textit{most foods are unnatural} appeal; Treat article \#3: \textit{social norm} appeal.}
    % \subcaption*{\small \textbf{Fig. \ref{fig:2A}.} }
\end{figure}
% ------------------------------------- %
\FloatBarrier

% ------------------------------------- %
% FIGURE:
\begin{figure}[h]
    \begin{center}
    \includegraphics[width=0.85\linewidth]{\figurepath{effects.png}}
    \end{center}
    \caption{\label{fig:effects}\footnotesize \textbf{Effects on interest in clean meat and concerns.} Points in black represent the difference in means between exposed and unexposed respondents on changes in interest in clean meat and concerns about clean meat between waves 1 and 3. Gray points represent the differences in means between waves 1 and 2. Horizontal bars represent 90\% and 95\% confidence intervals. Non-binary dependent variables are standardized to have mean equal to zero and variance equal to one. Treat article \#1: \textit{natural does not mean good} appeal; Treat article \#2: \textit{most foods are unnatural} appeal; Treat article \#3: \textit{social norm} appeal.}
    % \subcaption*{\small \textbf{Fig. \ref{fig:2A}.} }
\end{figure}
% ------------------------------------- %
\FloatBarrier





% APPEAL EFFECTS
% --------------
% reports main effects of message treatments (RQ1) and social information treatment (RQ2).
\subsection{Appeal effects}

Can simple pro-clean meat appeals aimed at debunking the naturalistic fallacy or conveying descriptive social norms effectively overcome the pernicious effects of negative social infomation shown above? Here, we show that, while pro-clean meat appeals based on debunking or conveying social norms can counteract nearly all of the negative effects of negative social information \textit{in the short term}, these counteracting effects decay over time and are indistinguishable from zero after 10 weeks.

Figure \ref{fig:effects} displays the effects of the \textit{natural does not mean good} appeal (Appeal \#1), \textit{most foods are unnatural} appeal (Appeal \#2), and \textit{social norm} appeal (Appeal \#3) on interest in clean meat and concerns about clean meat relative to placebo. Points in black represent the difference in means between exposed and unexposed respondents on changes in interest in clean meat and concerns about clean meat between waves 1 and 3. Gray points represent the differences in means between waves 1 and 2. 

% In both figures, outcome variables are measured in terms of the change between baseline and endline surveys. We discuss the main results below.

% EFFECTS ON WAVE 2
As shown in Panels 2-4 in Figure \ref{fig:effects}, all three pro-clean meat appeals led to improvements in consumer attitudes towards clean meat products -- as measured by \textit{would eat}, \textit{interest purchase}, and \textit{feel} -- between waves 1 and 2 relative to placebo. Both of the debunking appeals (\textit{natural does not mean good} and \textit{most foods are unnatural}) also led to decreases in concerns about whether clean meat products are unnatural. For instance, the proportion of respondents in the \textit{natural does not mean good} arm raising ``unnatural'' as a concern decreased by roughly 15 percentage points more between waves 1 and 2 than among respondents reading the placebo article. The \textit{social norms} appeal also led to a non-trivial reduction in the number of respondents raising the ``unnatural'' concern, yet this effect does not reach conventional levels of statistical significance.
% FOOTNOTE
\footnote{The \textit{most foods are unnatural} and \textit{social norms} may have also led to sizable reductions in concerns about the safety of clean meat, which should be expected given that the former emphasizes that clean meat is not very different from many other safe food products and the latter normalizes clean meat products by emphasizing widespread consumer excitement about their availability. However, these effects do not reach conventional levels of statistical significance.}
% END FOOTNOTE

% LACK OF PERSISTENCE
Overall, however, these effects did not persist over time. For instance, while, the \textit{social norms} appeal consistently had somewhat larger effects than the other two appeals on consumer interest in clean meat when measured at wave 2, these effects had completaly shrunk to zero by wave 3 (10 weeks after treatment exposure). The same pattern appears for the \textit{natural does not mean good} appeal, where the positive effects on interest in clean meat at wave 2 disappear when considering wave 3 measures of interest in clean meat. The \textit{most foods are unnatural} appeal provides some optimism, since respondents exposed to this appeal were still more likely to report increases in interest in clean meat after 10 weeks relative to the placebo group, yet these effects do not reach conventional levels of statistical significance. Finally, the anti-naturalistic fallacy effects of the two debunking appeals in wave 2 do not persist to wave 3.

% SUMMARY
In short, despite encouraging short-term effects of the pro-clean meat appeals, we fail to find evidence of persistence in these effects over a 10 week period. This lack of persistence is especially concerning given that the pernicious effects of negative social information continued to suppress interest in clean meat after 10 weeks.

% Yet, the effects on willingness to be notified when clean meat products become available (\textit{notified available} and \textit{entered email}) are significantly weaker.

% while each of the three pro-clean meat appeals examined in this study can largely -- but not completely -- wash out these negative effects. 

% % ------------------------------------- %
% % FIGURE:
% \begin{figure}[h]
%     \begin{center}
%     \includegraphics[width=0.7\linewidth]{\figurepath{effects_appeals_interest.png}}
%     \end{center}
%     \caption{\label{fig:effects_appeals_interest}\footnotesize \textbf{Appeal effects on interest in clean meat.} Each dot represents the estimated treatment effect of a pro-clean meat appeal (relative to the placebo appeal) on a given outcome measure. Outcome measures are computed as the change between baseline and treatment waves. Horizontal bars represent 90\% and 95\% confidence intervals. Treat article \#1: \textit{natural does not mean good} appeal; Treat article \#2: \textit{most foods are unnatural} appeal; Treat article \#3: \textit{social norm} appeal.}
%     % \subcaption*{\small \textbf{Fig. \ref{fig:2A}.} }
% \end{figure}
% % ------------------------------------- %
% \FloatBarrier

% CONCERNS ABOUT CLEAN MEAT
% \textbf{Concerns about clean meat.} Figure \ref{fig:effects_appeals_concerns} illustrates that the \textit{natural does not mean good} and \textit{most foods are unnatural} appeals effectively reduced consumer concerns about whether clean meat is ``unnatural'', as we should expect. 


% Finally, the three appeals had no demonstrable effects on consumer concerns about the health benefits, taste, or cost of clean meat products. Overall, the substantial effects of the pro-clean meat appeals on consumer concerns about clean meat's ``natural'' qualities provides reason for optimism that these appeals can effectively improve consumer attitudes by combating the naturalistic fallacy. However, the weak effects on consumer concerns about clean meat's safety and health benefits raise doubt about the degree to which these appeals can effectively induce other shifts in consumer concerns about clean meat products.

% Figures \ref{fig:interest}-\ref{fig:benefit} display the effects of the three appeals relative to placebo, alongside the effects of negative social information, using outcome measures collected in the treatment survey wave. Altough we are still awaiting the results of the endline survey, these results show promising evidence that all three appeals are moderately effective at improving public attitudes towards clean meat products. However, these promising effects are over-shadowed by the effects of negative social information, which greatly reduces respondents' acceptance of and perceived benefits of clean meat products.

% % ------------------------------------- %
% % FIGURE:
% \begin{figure}[h]
%     \begin{center}
%     \includegraphics[width=0.8\linewidth]{\figurepath{effects_appeals_concerns.png}}
%     \end{center}
%     \caption{\label{fig:effects_appeals_concerns}\footnotesize \textbf{Appeal effects on concerns about clean meat.} Each dot represents the estimated treatment effect of a pro-clean meat appeal (relative to the placebo appeal) on a given outcome measure. Outcome measures are computed as the change between baseline and treatment waves. Horizontal bars represent 90\% and 95\% confidence intervals. Treat article \#1: \textit{natural does not mean good} appeal; Treat article \#2: \textit{most foods are unnatural} appeal; Treat article \#3: \textit{social norm} appeal.}
%     % \subcaption*{\small \textbf{Fig. \ref{fig:2A}.} }
% \end{figure}
% % ------------------------------------- %
% \FloatBarrier


% \textbf{Perceptions of social norms.} \todo{...}

% \textbf{Perceptions of vegetarians.} \todo{...}

% \textbf{Perceptions of intelligence and sentience.} \todo{...}

% SUBGROUP VARIATION IN EFFECTIVENESS

\subsection{Did appeals influence the least interested respondents?}

Did the pro-clean meat appeals have more persistent effects on individuals who were already more/less interested in consuming clean meat products at baseline? Similarly, were the pernicious effects of negative social information concentrated among individuals who were already more/less interested in consuming clean meat products at baseline? We examine both of these questions in Figure \ref{fig:subgroup_effects_interest} (\nameref{sec:appendix}), displaying the estimated treatment effects of negative social information and the three pro-clean meat appeals on interest in clean meat measured 10 weeks after treatment exposure. The treatment effects are estimated separately for individuals who reported ``low'', ``neutral'', and ``high'' levels of interest in clean meat in the baseline survey (wave 1), measured based on responses to \textit{feel} (left column) and \textit{interest purchase} (right column). Looking at the bottom row of the figure, negative social information has little persistent effect on individuals who were already very interested in clean meat at baseline. Instead, negative social information undermines interest in clean meat among respondents with low to moderate levels of existing interest in clean meat products. On the one hand, this result is encouraging, suggesting that once consumers have been won over, then they may be more resilient to negative coverage of clean meat products. On the other hand, high levels of susceptibility to negative social information among consumers with low to moderate interest in clean meat adds to the difficulties of converting skeptical consumers into regular buyers of clean meat products.

Turning to the three pro-clean meat appeals (rows 1-3), there is some evidence that the \textit{most foods are unnatural} (Appeal \#2) led to persistent improvements in interest in clean meat after 10 weeks among respondents who reported being \textit{least} interested in clean meat at baseline. This result is encouraging, suggesting that a combination of debunking and analogies to common foods can be an effective strategy for overcoming skepticism from the least interested consumers. The other two appeals, however, fail to display any persistent effects regardless of respondents' baseline levels of interest in clean meat.

% While the results above show that none of the three pro-clean appeals have persistent positive effects,   If the appeals have little effect on those that are opposed to or undecided about clean meat, then the results reported above would greatly \textit{overstate} the potential for expanding the clean meat market through short pro-clean meat appeals of the sort examined in this experiment. Fortunately, as we show in Figure \ref{fig:subgroup_effects_interest} (\nameref{sec:appendix}), the effects of each pro-clean meat appeal on various measures of interest in clean meat are \textit{not} systematically smaller among individuals who reported being less interested in clean meat in the baseline survey.

% \footnote{We also examine whether individuals' recollection of clean meat products influences their susceptibility to negative social information and responsiveness to the treatment appeals (results not shown). We find that individuals who report remembering only a ``moderate amount'' or less about clean meat products at the beginning of the treatment survey wave are not differentially affected by negative social information or pro-clean meat appeals relative to individuals who claim to remember clean meat products very well.}

% Todo...
% In addition, Figure \ref{fig:subgroup_negative_agree} (\nameref{sec:appendix}) displays the effects of the three pro-clean meat appeals broken down by whether respondents agreed or disagreed with the negative social information (if shown). Respondents who \textit{disagreed} with the negative social information -- indicating favorable baseline attitudes towards clean meat -- were not affected by the pro-clean meat appeals. In contrast, respondents who \textit{agreed} with the negative social information were more likely to increase their acceptance of clean meat products after reading one of the three appeals.

% In short, \textit{we find no evidence} to suggest that the positive effects of the pro-clean meat appeals on consumer attitudes are concentrated among individuals who are already supportive of clean meat products.

% Rather, we find evidence to suggest that pro-clean meat appeals should be targeted towards individuals who are more susceptible to believing negative social information

\subsection{Did appeals influence the biggest meat-eaters?} 

For clean meat products to achieve significant environmental, animal welfare, and health impacts, they will need to be adopted by typical meat-eaters rather than vegetarians and ``reducetarians'' who only eat a few servings of meat products each week. In Figure \ref{fig:subgroup_effects_meat} (\nameref{sec:appendix}), we estimate the effects of negative social information and the three pro-clean meat appeals separately for individuals who reported eating 0-10 servings of meat per week at baseline, 10.5-17 servings of meat per week at baseline, and more than 17 servings of meat per week at baseline. Curiously, negative social information has weaker effects on interest in clean meat after 10 weeks among the top third of meat eaters (more than 17 servings), yet these differences are not consistent for all measures of interest in clean meat. Conversely, each of the three pro-clean meat appeals is marginally more effective at improving interest in clean meat among the bottom third of meat eaters (0-10 servings), although the differences in effectiveness are imprecisely estimated.

% Here, we find that the pro-clean meat appeals had roughly similar effects across all four levels of weekly meat consumption. If anything, the effects are smaller among individuals who reported eating \textit{less} meat at baseline.

% In addition, we do not find any evidence that individuals who eat vegetarian meats -- such as tofu and tempeh -- responded more positively to the appeals, adding further optimism that the pro-clean meat appeals have the potential to encourage meat-eaters to substitute clean meat products in place of conventional meat products.


% Third, are effects driven by individuals who are already sympathetic to the message of reducing conventional meat consumption?


% Finally, we examine whether message effectiveness varies by subgroup.

% \todo{...}

% \textbf{Initial feelings towards clean meat.} First, individuals who themselves bring up the naturalistic fallacy at baseline or who state that they agree with the negative social information treatment are of particular interest to us. We test whether these individuals are more or less responsive to the anti-naturalistic fallacy treatments.

% \textbf{Diet choice priorities.} We expect that individuals who state at baseline that they care most about nutritional information and health effects when choosing between food products will be less receptive to the anti-naturalistic fallacy treatments. We also expect that individuals who express higher levels of baseline concern for the environment or animal well-being will be more receptive to the anti-naturalistic fallacy treatments.

% \textbf{Utilitarian index.} As pre-specified in a companion paper, we construct a measure of subjects' Utilitarianism. We expect that individuals who score higher on this measure will be more receptive to the anti-naturalistic fallacy treatments in general but less receptive to the social norms anti-naturalistic fallacy treatment.

% \textbf{Demographics.} Finally, we examine whether treatment effects vary systematically by age and gender -- two demographic characteristics that are easy for advertisers to target.


% INTERACTIONS BETWEEN TREATMENTS
% -------------------------------
% \subsection{Overcoming negative social information}
% RQ3: What kinds of messages are most effective at combating the naturalistic fallacy and increasing individual acceptance of clean meat in the face of negative social information about clean meat?

\subsection{Combating negative social information}

As shown in Figure \ref{fig:effects}, the pro-clean meat appeals counteract much of the pernicious effects of negative social information in the short-term, but not over longer periods. However, these patterns do not shed light on whether any of the appeals \textit{interact} with the negative social information -- that is, whether any of the three appeals become significantly more or less effective when preceded by negative social information. We examine this possibility in Figure \ref{fig:subgroup_effects_negative} (\nameref{sec:appendix}), displaying the effects of each appeal on interest in clean meat products after 10 weeks, broken down by whether respondents were exposed to negative social information or not. Points in black represent the estimated appeal effects among respondents exposed to negative social information, while gray points represent estimated appeal effects among respondents not exposed.

There is some evidence that the \textit{natural does not mean good} appeal (Appeal \#1) is less effective at increasing interest in clean meat relative to placebo when preceded by negative social information. Overall, however, the black and gray points track each other closely, suggesting that in the aggregate none of the three appeals are particularly ill- or well-suited for combatting negative social information.

% Conversely, the point estimates for the \textit{most foods are unnatural} appeal (Appeal \#2) remain positive (although less precisely estimated) effects of the 


% Here, the \textit{most foods are unnatural} and \textit{social norms} appeals are just as effective at improving consumer attitudes towards clean meat whether or not respondents had just been exposed to negative social information. In contrast, the \textit{natural does not mean good} appeal is marginally \textit{more} effective in the face of negative social information, with all coefficients somewhat larger when respondents had just read anti-clean meat quotes. However, these estimates are imprecise and should not be given much inferential weight. Hence, overall we find that all three pro-clean meat appeals are roughly as effective in the face of negative social information, and there is no single appeal that is consistently more effective than the others in this setting.

 % while the \textit{natural does not mean good} appeal becomes \textit{more} effective at increasing consumer acceptance of clean meat products in the face of negative social information.

% Here, we see that the \textit{social norms} appeal is most effective at increasing consumer acceptance of clean meat products whether respondents are exposed to negative social information or not. 

% While it is promising that, for the most part, the treatment appeals continue to have positive effects on consumer acceptance of clean meat products in the face of negative social information, Figure \ref{fig:effects_social} contains a less encouraging result: \textit{none} of the three pro-clean meat appeals entirely counteract the negative effects of social information. On average, respondents who read the \textit{placebo} appeal and who were \textit{not} exposed to negative social information maintained a higher level of acceptance of clean meat products than respondents who were exposed to negative social information in combination with any one of the three pro-clean meat appeals. Hence, in order to overcome the pernicious effects of negative social information, individuals are likely to need repeated exposure to pro-clean meat appeals.


% In terms of interest in clean meat (Figures \ref{fig:interact_interest_purchase} and \ref{fig:interact_would_eat}), the \textit{most foods are unnatural} and \textit{social norms} articles did little to counteract the effects of negative social information. More optimistically, the effects of negative social information are somewhat smaller under the \textit{natural does not mean good} relative to the \textit{placebo} article, yet these differences are not significant.

% Turning to concerns about clean meat, negative social information leads respondents to raise ``unsafe'' as a concern about clean meat much more often after reading the \textit{natural does not mean good} article, while negative social information has little effect on concerns about the ``unsafe'' qualities of clean meat after reading the \textit{social norms} article (Figure \ref{fig:interact_concern_unsafe}). The effects of negative social information on propensity to raise the concern that clean meat is ``unnatural'' does not vary substantially by treatment article (Figure \ref{fig:interact_concern_unnatural}).

% \note{Not sure these are the visualization we want (and they take up way too much space anyways). We may be more interested in seeing how the positive effects of each of the four articles varies according to negative social information.}

% \note{In these four figures below, x-axis is exposure to negative social information, and the facetting variable is the four articles.}

% % ------------------------------------- %
% % FIGURE:
% \begin{figure}[h]
% \begin{subfigure}{0.48\textwidth}
%     \centering
%     \subcaption{Would eat}
%     \includegraphics[width=1\linewidth]{\figurepath{interact_means_would_eat.png}}
% \end{subfigure}
% \begin{subfigure}{0.48\textwidth}
%     \centering
%     \subcaption{Interest in purchasing}
%     \includegraphics[width=1\linewidth]{\figurepath{interact_means_interest_purchase.png}}
% \end{subfigure}
% \caption{\small\label{fig:interact_means}\textbf{Appeal effects by exposure to negative social information.} These plots display the averages for respondents exposed to each of the treatment appeals (x axis) in terms of whether they would eat clean meat products (subplot (a)) and whether they are interested in purchasing clean meat products (subplot (b)), broken down by whether respondents were exposed to negative social information or not. 90\% and 95\% confidence intervals are shown.}
% \end{figure}
% % ------------------------------------- %


% % ------------------------------------- %
% % FIGURE:
% \begin{figure}[h]
%     \begin{center}
%     \includegraphics[width=0.6\linewidth]{\figurepath{interact_interest_purchase.png}}
%     \end{center}
%     \caption{\label{fig:interact_interest_purchase}\small Effects of negative social information on interest in purchasing clean meat (y-axis) by treatment appeal. Point estimates represent group means. Exposure to negative social information is displayed on the x-axis (0=no exposure; 1=exposure). The four facet plots correspond to the four appeals (0=placebo appeal; 1=\textit{natural does not mean good} appeal; 2=\textit{most foods are unnatural} appeal; 3=\textit{social norms} appeal). Vertical bars represent 95\% confidence intervals.}
%     % \subcaption*{\small \textbf{Fig. \ref{fig:2A}.} }
% \end{figure}
% % ------------------------------------- %
% \FloatBarrier



% WHO IS SUSCEPTIBLE TO THE NATURALISTIC FALLACY?
% -----------------------------------------------
% \subsection{Susceptibility to the naturalistic fallacy}


% \todo{Do these analyses. In this subsection, we examine the baseline predictors of who raises concerns about the naturalistic fallacy. First, we examine predictors at baseline only (i.e. before treatment assignment). Second, we examine whether negative social information led certain types of individuals to be more likely to bring up ``unnatural'' as a concern about cultured meat.}

% Absent of any messaging or social information, what kinds of individuals are most susceptible to bringing up the naturalistic fallacy after learning about clean meat?

% \begin{itemize}
%     \item H4.1: individuals with lower levels of education are more likely to state “unnatural” as a concern about clean meat and to be less accepting of clean meat overall.
%     \item H4.2: individuals with lower levels of household income are more likely to state “unnatural” as a concern about clean meat and to be less accepting of clean meat overall.
%     \item H4.3: older individuals are more likely to state ``unnatural'' as a concern about clean meat and to be less accepting of clean meat overall.
%     \item H4.4: individuals with a more conservative political leaning are more likely to state “unnatural” as a concern about clean meat and to be less accepting of clean meat overall.
%     \item H4.5: individuals who score higher on a scale of Utilitarianism are less likely to state "unnatural" as a concern about clean meat and to be more accepting of clean meat overall.
%     \item H4.6: individuals who state at baseline that they care most about nutritional information and health effects when choosing between food products will be more likely to raise “unnatural” as a concern about clean meat.
% \end{itemize}


% --------------------------------------------------------------------- %
% --------------------------------------------------------------------- %
%                             SECTION                                   %
% --------------------------------------------------------------------- %
% --------------------------------------------------------------------- %


% \section{Discussion}
% \label{sec:discussion}

% \todo{...}

% --------------------------------------------------------------------- %
% --------------------------------------------------------------------- %
%                       CONCLUDING REMARKS                              %
% --------------------------------------------------------------------- %
% --------------------------------------------------------------------- %


\section{Concluding remarks}
\label{sec:conclusion}

Overall, we show that simple ``debunking'' appeals aimed at combating the naturalistic heuristic can be effective at improving public acceptance of cultured meat products. At the same time, however, these positive effects can easily be over-shadowed by small amounts of negative social information. More results to come soon.

% --------------------------------------------------------------------- %
% --------------------------------------------------------------------- %
%                               APPENDIX                                %
% --------------------------------------------------------------------- %
% --------------------------------------------------------------------- %

\section*{Appendix}
\label{sec:appendix}

% ------------------------------------- %
% FIGURE:
\begin{figure}[h]
    \begin{center}
    \includegraphics[width=0.9\linewidth]{\figurepath{interest.png}}
    \end{center}
    \caption{\label{fig:interest}\footnotesize \textbf{Interest in clean meat products.}}
    % \subcaption*{\small \textbf{Fig. \ref{fig:2A}.} }
\end{figure}
% ------------------------------------- %
\FloatBarrier

% ------------------------------------- %
% FIGURE:
\begin{figure}[h]
    \begin{center}
    \includegraphics[width=0.85\linewidth]{\figurepath{subgroup_effects_interest_3.png}}
    \end{center}
    \caption{\label{fig:subgroup_effects_interest}\footnotesize \textbf{Heterogeneity in appeal effects by baseline interest in clean meat.} Horizontal bars represent 90\% and 95\% confidence intervals. Non-binary dependent variables are standardized to have mean equal to zero and variance equal to one. Treat article \#1: \textit{natural does not mean good} appeal; Treat article \#2: \textit{most foods are unnatural} appeal; Treat article \#3: \textit{social norm} appeal.}
    % \subcaption*{\small \textbf{Fig. \ref{fig:2A}.} }
\end{figure}
% ------------------------------------- %
\FloatBarrier


% ------------------------------------- %
% FIGURE:
\begin{figure}[h]
    \begin{center}
    \includegraphics[width=0.7\linewidth]{\figurepath{subgroup_effects_meat_chg_1_3_std.png}}
    \end{center}
    \caption{\label{fig:subgroup_effects_meat}\footnotesize \textbf{Heterogeneity in appeal effects by number of servings of meat consumed per week at baseline.} Horizontal bars represent 90\% and 95\% confidence intervals. Non-binary dependent variables are standardized to have mean equal to zero and variance equal to one. Treat article \#1: \textit{natural does not mean good} appeal; Treat article \#2: \textit{most foods are unnatural} appeal; Treat article \#3: \textit{social norm} appeal.}
    % \subcaption*{\small \textbf{Fig. \ref{fig:2A}.} }
\end{figure}
% ------------------------------------- %
\FloatBarrier

% % ------------------------------------- %
% % FIGURE:
% \begin{figure}[h]
% \begin{subfigure}{0.48\textwidth}
%     \centering
%     \subcaption{Would eat}
%     \includegraphics[width=1\linewidth]{\figurepath{subgroup_negative_agree_would_eat.png}}
% \end{subfigure}
% \begin{subfigure}{0.48\textwidth}
%     \centering
%     \subcaption{Interest in purchasing}
%     \includegraphics[width=1\linewidth]{\figurepath{subgroup_negative_agree_interest_purchase.png}}
% \end{subfigure}
% \caption{\small\label{fig:subgroup_negative_agree}\textbf{Appeal effects by agreeance with negative social information.} These plots display the averages for individuals exposed to each of the treatment appeals (x axis) in terms of whether they would eat clean meat products (subplot (a)) and whether they are interested in purchasing clean meat products (subplot (b)), broken down by whether respondents agreed or disagreed with the negative social information.  90\% and 95\% confidence intervals are shown.}
% \end{figure}
% % ------------------------------------- %

% ------------------------------------- %
% FIGURE:
\begin{figure}[h]
    \begin{center}
    \includegraphics[width=0.7\linewidth]{\figurepath{subgroup_effects_negative_chg_1_3.png}}
    \end{center}
    \caption{\label{fig:subgroup_effects_negative}\footnotesize \textbf{Heterogeneity in appeal effects by whether respondents exposed to negative social information.} Horizontal bars represent 90\% and 95\% confidence intervals. Non-binary dependent variables are standardized to have mean equal to zero and variance equal to one. Treat article \#1: \textit{natural does not mean good} appeal; Treat article \#2: \textit{most foods are unnatural} appeal; Treat article \#3: \textit{social norm} appeal.}
    % \subcaption*{\small \textbf{Fig. \ref{fig:2A}.} }
\end{figure}
% ------------------------------------- %
\FloatBarrier

% --------------------------------------------------------------------- %
% --------------------------------------------------------------------- %
%                             REFERENCES                                %
% --------------------------------------------------------------------- %
% --------------------------------------------------------------------- %
% \newpage
\bibliographystyle{chicago}
\bibliography{references}

\end{document}
